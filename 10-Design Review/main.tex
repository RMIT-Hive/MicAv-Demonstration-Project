\documentclass{article}
% Package Import

\usepackage[a4paper, margin=2.2cm, headheight=2cm, headsep=0.6cm, voffset=0.1cm]{geometry} %Document parameters tweaked so text fits good :)
\usepackage{natbib}                     % Bilbiography / Referencing Library
\usepackage{multicol}                   % Enables Multiple columns
\usepackage{graphicx}                   % Allows inclusion
\usepackage{subcaption}                 % Enables support for writing subcaptions
\usepackage[export]{adjustbox}          % Allows for better adjustment of figure box sizing
\usepackage{wrapfig}                    % Wrap around text around figures
\usepackage{float}                      % Allows stable movement of floating tables and text around tables
\usepackage{sectsty}                    % Allows Section Style configuration
\usepackage{xcolor,colortbl}            % Colour package... everything spelt in American English >_<
\usepackage{fancyhdr}                   % Easy Footers and Headers
\usepackage{fontspec}                   % Importable fonts
\usepackage{tabularx}                   % Easily comfigureable tables and table sizes
\usepackage{enumitem}                   % Easily configureable lists and custom lists
\usepackage[hidelinks]{hyperref}        % Allows clickable hyperlinks - is primarily used for table of contents

    
% Font Definition

\newfontfamily{\eco}{Economica}[Path = ../00-Formatting/, Extension = .ttf, UprightFont = *-Regular, BoldFont= *-Bold, ItalicFont = *-Italic, BoldItalicFont = *-BoldItalic]
\newfontfamily{\dmsans}{DMSans}[Path = ../00-Formatting/, Extension = .ttf, UprightFont = *-Regular, BoldFont= *-Bold, ItalicFont = *-Italic, BoldItalicFont = *-BoldItalic]
\setmainfont{DMSans}[Path = ../00-Formatting/, Extension = .ttf, UprightFont = *-Regular, BoldFont= *-Bold, ItalicFont = *-Italic, BoldItalicFont = *-BoldItalic]


% Section Style Commands

\sectionfont{\eco\uppercase}
\subsectionfont{\eco\uppercase}
\subsubsectionfont{\eco\uppercase}
\urlstyle{same}


% Document Headers and Footers

\pagestyle{fancy}
\fancyhf{}
\fancyhead[L]{\includegraphics[width=0.07\textwidth]{00-Formatting/logo_light.png}}
\fancyfoot[R]{\eco\\ \thepage}


% Setup a Custom List Style

\newlist{regulation}{enumerate}{1}
\setlist[regulation]{label=(\arabic*)}

% Miscellaneous Document Rules

\renewcommand{\footrulewidth}{0.4pt} % default is 0pt
\bibliographystyle{plain} % Normal command for Natbib style - this style is IEEE (basically)
\renewcommand{\bibsection}{\section{Bibliography}} % This sets the bibliography to be it's own section (titled 'Bibliography')

%%%%%%%%%%%%%%%%%%%%%%%%%%%%%%%%%%%%
%%% DOCUMENT EDIT AREA **START** %%%
%%%%%%%%%%%%%%%%%%%%%%%%%%%%%%%%%%%%

\newcommand{\documentCategory}{R\&T Project} % Is this document an R&T Programme / Project (for example)? 
\newcommand{\documentType}{Design Review} % is this document a Formulation Authorisation Document (FAD)? A Research Report? Etc.
\newcommand{\documentTitle}{Micro Avionics Unit} % this is the specific name of the report subject matter
\newcommand{\documentTitleShortHand}{MicAv} % this is a short-hand reference to the subject matter (if one exists)

\newcolumntype{a}{>{\columncolor{lightgray}}l} % If you put 'a' instead of 'c' in the tabular structure, the result will be a light-grey column

\begin{document}
\begin{titlepage}
    \begin{center}
        \vspace*{1cm}
        
        \huge{\eco High Velocity Research}\\
        \huge{\eco \documentCategory}\\
        \vspace{0.8cm}
        \large{\eco \documentType}\\
        \vspace{0.8cm}
        \large{\eco \documentTitle}\\
        \textit{\eco \documentTitleShortHand}\\
        \vspace{3.5cm}
        \includegraphics[width=0.2\textwidth]{00-Formatting/logo_light.png}
    
        \vspace{2.5cm}
    
    \end{center}
\end{titlepage}

\tableofcontents
\pagebreak

%%%%%%%%%%%%%%%%%%%%%%
%%% REPORT CONTENT %%%
%%%%%%%%%%%%%%%%%%%%%%

\section{Introduction}

    This report outlines the design, prototyping, testing and validation for a basic passive avionics unit.
    The avionics unit is outfitted with a pressure sensor that can be used to calculate the altitude of the aeronautical system it is integrated to.

\section{Requirements}

    The requirements of this avionics unit are as follows:
    \begin{itemize}
        \item Unit shall possess a datalogging ('telemetry') capability:
        \begin{itemize}
            \item Telemetry shall record altitude with respect to time.
            \item Telemetry shall store recorded data in nonvolatile memory.
            \item Telemetry shall perform in "L1" unguided rocket flight conditions, including:
            \begin{itemize}
                \item Up to 40 g of acceleration
                \item Up to 340 m/s of velocity
                \item Up to 1,500 m of altitude
            \end{itemize}
        \end{itemize}
    \end{itemize}

    Altitude data is usually the most interesting bit of flight data for a small unguided rocket, as it will validate the trajectory simulation used to design the rocket.
    The avionics unit will store the altitude data to be examined once the unit is recovered;
    This might equate to an analysis of the maximum altitude of a rocket after it has been recovered.
    The altitude data must be in nonvolatile memory so that if the avionics unit runs out of power the data is not lost.
    Otherwise, this altitude data may be recorded and stored in whatever implementation (i.e. method) works.

\section{Detailed Design}

    \subsection{Altitude Measurement}

        There are two types of sensors that can be used for measuring the altitude of the unit; a GPS or a pressure sensor.

        The Global Position System works by triangulating the position of a GPS unit using the Global Navigation Satellite System, by measuring the errors in transmission/reception of GNSS signals.
        A GPS is able to measure altitude as well as position on the surface of the Earth, as long as it has four or more sattelites acquired.
        GPS units have an in-built acceleration and velocity limit (GPS "CoCom" limits) so they cannot be used in guided missile systems.
        The CoCom limit is typically quoted as applying to systems above:
        \begin{itemize}
            \item >60,000 ft (>18.3 km)
            \item >1,000 knots (>514 m/s)
        \end{itemize}
        GPS systems may be specified for greater limits (e.g. U-Blox NEO-6 is specified to operate up to 50km \cite{ublox-gps1}) but it should be assumed that the CoCom limits be applied to all GPS units above the altitude OR velocity limit unless otherwise tested.

        A small number of low-cost (<AUD 60 for breakout) GPS units were examined;
        \begin{itemize}
            \item CD-PA1616S GPS patch antenna module (sold in breakouts by Adafruit) \cite{cd-gps}.
            \item NEO-6 U-blox 6 GPS module (sold in breakouts by U-Blox) \cite{ublox-gps1}.
            \item SAM-M8Q U-blox M8 GNSS antenna module (sold in breakouts by SparkFun) \cite{ublox-gps2}.
        \end{itemize}

        The CoCom limits are not too much for a small ('L1') unguided rocket, but the acceleration limit is.
        These three GPS sensors have an acceration limit of 4 g or greater.
        If this limit is exceeded, the GPS may be disabled for a period of time.
        Therefore, a GPS likely is not an acceptable method for altitude measurement in a rocket.

\section{Technical Development}

    The initial prototype system

%%%%%%%%%%%%%%%%%%%%%%%%%%%%%%%%%%%
%%% DOCUMENT EDIT AREA **STOP** %%%
%%%%%%%%%%%%%%%%%%%%%%%%%%%%%%%%%%%

\pagebreak
\appendix

\section{Acronyms}
\begin{itemize}
    \item \textbf{ACRONYM} - Acronym fully expanded 
\end{itemize}

\section{Definitions}
\begin{itemize}
    \item \textbf{Word} - Definition for 'Word' 
\end{itemize}

\bibliography{01-Sections/citations}
% Use 'Natbib' library for referencing, styles are automatically configured, simply add reference information in the "01-Sections/citations" area.
% This will be auto generated when a citation is entered
\end{document}